\documentclass[11pt, a4paper]{article}


\newcommand{\FirstName}{Leonardo}
\newcommand{\LastName}{Uieda}
\newcommand{\MyName}{\FirstName\ \LastName}
\newcommand{\Title}{Curriculum Vit\ae}
\newcommand{\Email}{leouieda@gmail.com}
\newcommand{\TablePad}{\vspace{-0.4cm}}
% Macros to add links to code and PDF versions of articles
\newcommand{\Code}[1]{\href{#1}{\fontsize{10}{0}\selectfont [code]}}
\newcommand{\PDF}[1]{\href{#1}{\fontsize{10}{0}\selectfont [pdf]}}
% Macros to set the year and duration
\newcommand{\Duration}[2]{\fontsize{10}{0}\selectfont #1\ --\ #2}
\newcommand{\Year}[1]{\fontsize{10}{0}\selectfont #1}


\usepackage{graphicx}
\usepackage{tabularx}
% For multipage tables
\usepackage{ltablex}

% Define command to insert month name and year as date
\usepackage{datetime}
\newdateformat{monthyear}{\monthname[\THEMONTH], \THEYEAR}

% Set the page margins
\usepackage[left=0.7in,right=0.7in,top=0.7in,bottom=1in]{geometry}

% No indentation
\setlength\parindent{0cm}

% Increase the line spacing
\renewcommand{\baselinestretch}{1.1}
\renewcommand{\arraystretch}{1.3}

% Remove space between items in itemize and enumerate
\usepackage{enumitem}
\setlist{nosep}

% Use custom colors
\usepackage[usenames,dvipsnames]{xcolor}

% Set fonts. Requires compilation with xelatex
\usepackage{fontspec}
\setmainfont[BoldFont=SourceSansPro-Semibold]{SourceSansPro-Light}
\setmonofont{Source Code Pro}
% Configure the font style for sections
\usepackage{sectsty}
\sectionfont{\vspace{0.2cm}\mdseries\fontsize{16}{0}\selectfont\uppercase}
\subsectionfont{\mdseries\uppercase}
% Control the font size
\usepackage{anyfontsize}

% Set headers
%\usepackage{fancyhdr}
%\pagestyle{fancy}
%\fancyhf{}
%\chead{\MyName \hspace{0.2cm} | \hspace{0.2cm} \Title}
%\cfoot{\thepage}

% Metadata for the PDF output and control of hyperlinks
\usepackage[colorlinks=true]{hyperref}
\hypersetup{
    pdftitle={\MyName\ - \Title},
    pdfauthor={\MyName},
    linkcolor=blue,
    citecolor=blue,
    filecolor=black,
    urlcolor=MidnightBlue
}



\begin{document}


% No header for the first page
%\thispagestyle{plain}

\begin{tabular}{@{}l c@{}}
    \parbox{0.6\textwidth}{
        {\fontsize{36pt}{0}\selectfont \MyName}
        \\[0.5cm]
        {\fontsize{13pt}{0}\selectfont \Title \, | \, \monthyear\today}
    } &
    \parbox{0.4\textwidth}{
        \fontsize{10pt}{12pt}\selectfont
        Department of Geology and Geophysics, SOEST
        \\
        University of Hawai'i at Manoa
        \\
        1680 East-West Rd, POST 804
        \\
        96822 Honolulu, HI, USA
        \\
        \href{mailto:\Email}{\Email}
        \, | \,
        \href{http://www.leouieda.com}{www.leouieda.com}
    }
\end{tabular}

\vspace{0.5cm}


%%%%%%%%%%%%%%%%%%%%%%%%%%%%%%%%%%%%%%%%%%%%%%%%%%%%%%%%%%%%%%%%%%%%%%%%%%%%%%%
\section*{Research interests}

\TablePad
\begin{tabularx}{\textwidth}{rr}
    \parbox{0.5\textwidth}{
        \begin{itemize}
        \item Gravity and magnetics
        \item Inverse problem theory
        \end{itemize}
    }
    &
    \parbox{0.5\textwidth}{
        \begin{itemize}
        \item Open-source scientific software
        \item Reproducibility of computations
        \end{itemize}
    }
\end{tabularx}


%%%%%%%%%%%%%%%%%%%%%%%%%%%%%%%%%%%%%%%%%%%%%%%%%%%%%%%%%%%%%%%%%%%%%%%%%%%%%%%
\section*{Professional experience}

\TablePad
\begin{tabularx}{\textwidth}{@{}l X}
    \Duration{02/2017}{present}  &
    \textbf{Visiting Research Scholar} (postdoc),
    University of Hawaii at Manoa, USA
    \newline
    \textit{Working with Prof. Paul Wessel to build a Python interface for
    the Generic Mapping Tools}
    \\
    \Duration{02/2014}{present}  &
    \textbf{Assistant Professor},
    Universidade do Estado do Rio de Janeiro, Brazil
    \newline
    \textit{Professor of Geophysics at the Department of Applied Geology.
    Responsible for the Laboratory of Exploration Geophysics (LAGEX).}
    \\
    \Duration{02/2011}{03/2011}  &
    \textbf{Visiting Researcher},
    University of Trieste, Italy
    \newline
    \textit{Working with professor Carla Braitenberg to develop version 1.0 of
    the open-source software Tesseroids.}
\end{tabularx}


%%%%%%%%%%%%%%%%%%%%%%%%%%%%%%%%%%%%%%%%%%%%%%%%%%%%%%%%%%%%%%%%%%%%%%%%%%%%%%%
\section*{Education}

\TablePad
\begin{tabularx}{\textwidth}{@{}l X}
    \Duration{11/2011}{04/2016}  &
    \textbf{PhD in Geophysics}, Observatório Nacional, Brazil
    \newline
    Thesis: \textit{Forward modeling and inversion of gravitational fields in
    spherical coordinates}
    \newline
    \Code{https://github.com/leouieda/phd-thesis}
    \PDF{http://www.leouieda.com/about/phd.html}
    \\
    \Duration{03/2010}{10/2011}  &
    \textbf{MSc in Geophysics}, Observatório Nacional, Brazil
    \newline
    Thesis: \textit{Robust 3D gravity gradient inversion by planting anomalous
    densities}
    \newline
    \Code{https://github.com/pinga-lab/paper-planting-densities}
    \PDF{http://www.leouieda.com/about/masters.html}
    \\
    \Duration{08/2008}{05/2009}  &
    \textbf{International Exchange}, York University, Canada
    \\
    \Duration{03/2004}{12/2009}  &
    \textbf{BSc in Geophysics}, Universidade de São Paulo, Brazil
    \newline
    Thesis: \textit{Forward modeling of the gravity gradient tensor using
    tesseroids}
    \newline
    \Code{https://github.com/leouieda/barchelor-thesis}
    \PDF{http://www.leouieda.com/about/bachelors.html}
\end{tabularx}


%%%%%%%%%%%%%%%%%%%%%%%%%%%%%%%%%%%%%%%%%%%%%%%%%%%%%%%%%%%%%%%%%%%%%%%%%%%%%%%
\section*{Funding}

\TablePad
\begin{tabularx}{\textwidth}{@{}l X}
    \Duration{11/2014}{11/2017}  &
    QUALITEC/UERJ program for training a technician for the Laboratory of
    Exploration Geophysics (LAGEX). Total BRL \$154,800 as a scholarship for
    the technician.
    \\
    \Duration{11/2014}{11/2017}  &
    CAPES PhD
    \\
    \Duration{11/2014}{11/2017}  &
    CAPES MSc
    \\
    \Duration{11/2014}{11/2017}  &
    SBGf
    \\
    \Duration{11/2014}{11/2017}  &
    FAPESP
\end{tabularx}


%%%%%%%%%%%%%%%%%%%%%%%%%%%%%%%%%%%%%%%%%%%%%%%%%%%%%%%%%%%%%%%%%%%%%%%%%%%%%%%
\section*{Honors \& Awards}

\TablePad
\begin{tabularx}{\textwidth}{@{}l p{0.9\textwidth}}
    \Year{2017}  &
    SBGf
    \\
    ~ & meh
    \\
    \Year{2016}  &
    Paraninfo
\end{tabularx}


%%%%%%%%%%%%%%%%%%%%%%%%%%%%%%%%%%%%%%%%%%%%%%%%%%%%%%%%%%%%%%%%%%%%%%%%%%%%%%%
\section*{Open-source Software}

\TablePad
\begin{tabularx}{\textwidth}{@{}p{2.3cm} X}
    \raisebox{-1.9cm}{\includegraphics[width=2cm]{figures/fatiando.png}}  &
    \textbf{Fatiando a Terra}
    (\href{http://www.fatiando.org}{www.fatiando.org})
    \newline
    \emph{A Python library for modeling and inversion in geophysics.}
    \newline
    Provides tools for geophysical data analysis (mainly potential fields),
    forward modeling, and developing new inversion methods.
    \newline
    Source-code:
    \href{https://github.com/fatiando/fatiando}{github.com/fatiando/fatiando}
    \\
    \raisebox{-1.9cm}{\includegraphics[width=2cm]{figures/tesseroids.png}} &
    \textbf{Tesseroids}
    (\href{http://tesseroids.leouieda.com}{tesseroids.leouieda.com})
    \newline
    \emph{Forward modeling of gravitational fields in spherical coordinates.}
    \newline
    Command-line programs written in C for modeling the gravitational
    potential, acceleration, and gradient tensor in Cartesian and spherical
    coordinates.
    \newline
    Source-code:
    \href{https://github.com/leouieda/tesseroids}{github.com/leouieda/tesseroids}
\end{tabularx}


\end{document}
