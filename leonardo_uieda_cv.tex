\documentclass[11pt, a4paper]{article}

\newcommand{\FirstName}{Leonardo}
\newcommand{\LastName}{Uieda}
\newcommand{\Initials}{L}
\newcommand{\MyName}{\FirstName\ \LastName}
\newcommand{\Title}{Curriculum Vit\ae}
\newcommand{\Email}{leouieda@gmail.com}
\newcommand{\ORCID}{0000-0001-6123-9515}
\newcommand{\TablePad}{\vspace{-0.4cm}}
%\newcommand{\Item}{$\bullet$ \hspace{0.1cm}}
\newcommand{\Item}{}
\newcommand{\SoftwareTitle}[1]{{\fontsize{13pt}{0}\selectfont \bfseries #1}}
\newcommand{\TableTitle}[1]{{\fontsize{14pt}{0}\selectfont \itshape #1}}
\newcommand{\Invited}{[\textbf{invited}]}
% Macros to add links to code and PDF versions of articles
\newcommand{\Code}[1]{[\href{#1}{code}]}
\newcommand{\PDF}[1]{[\href{#1}{pdf}]}
\newcommand{\Slides}[1]{[\href{#1}{slides}]}
\newcommand{\Poster}[1]{[\href{#1}{poster}]}
\newcommand{\DOI}[1]{doi:\href{https://doi.org/#1}{#1}}
\newcommand{\Youtube}[1]{[\textbf{recording:} \href{https://youtu.be/#1}{youtu.be/#1}]}
% Macros to set the year and duration
\newcommand{\Duration}[2]{\fontsize{10pt}{0}\selectfont #1\ --\ #2}
\newcommand{\Year}[1]{\fontsize{10pt}{0}\selectfont #1}

% Names for citing coauthors
\newcommand{\Me}{\textbf{\LastName, \Initials}}
\newcommand{\Val}{Barbosa, VCF}
\newcommand{\Bi}{Oliveira Jr, VC}
\newcommand{\Paul}{Wessel, P}
\newcommand{\Carla}{Braitenberg, C}
\newcommand{\JB}{Silva, JBC}
\newcommand{\Dai}{Sales, DP}
\newcommand{\Figura}{Melo, FF}
\newcommand{\Dio}{Carlos, DU}
\newcommand{\BragaVale}{Braga, MA}
\newcommand{\YLi}{Li, Y}
\newcommand{\Angeli}{Angeli, G}
\newcommand{\Peres}{Peres, G}
\newcommand{\Everton}{Bomfim, EP}
\newcommand{\Eder}{Molina, E}
\newcommand{\Gomes}{Gomes, AAS}

\usepackage{graphicx}
\usepackage{tabularx}
% For multipage tables
\usepackage{ltablex}

% Define command to insert month name and year as date
\usepackage{datetime}
\newdateformat{monthyear}{\monthname[\THEMONTH], \THEYEAR}

% Set the page margins
\usepackage[left=1in,right=1in,top=1in,bottom=1in]{geometry}

% No indentation
\setlength\parindent{0cm}

% Increase the line spacing
\renewcommand{\baselinestretch}{1.1}
\renewcommand{\arraystretch}{1.5}

% Remove space between items in itemize and enumerate
\usepackage{enumitem}
\setlist{nosep}

% Use custom colors
\usepackage[usenames,dvipsnames]{xcolor}

% Set fonts. Requires compilation with xelatex
\usepackage{fontspec}
\setmainfont[BoldFont=SourceSansPro-Semibold]{SourceSansPro-Light}
\setmonofont{Source Code Pro}
% Configure the font style for sections
\usepackage{sectsty}
\sectionfont{\vspace{0.2cm}\mdseries\fontsize{16pt}{0}\selectfont\uppercase}
\subsectionfont{\itshape\mdseries\fontsize{14pt}{0}\selectfont}
% Control the font size
\usepackage{anyfontsize}

% Set headers
\usepackage{fancyhdr}
\pagestyle{fancy}
\fancyhf{}
\chead{
    \itshape
    \fontsize{10pt}{12pt}\selectfont
    \MyName
    \hspace{0.2cm} - \hspace{0.2cm}
    \Title
    \hspace{0.2cm} - \hspace{0.2cm}
    \monthyear\today
}
\rhead{}
\cfoot{\fontsize{10pt}{0}\selectfont \thepage}
\renewcommand{\headrulewidth}{0pt}

% Metadata for the PDF output and control of hyperlinks
\usepackage[colorlinks=true]{hyperref}
\hypersetup{
    pdftitle={\MyName\ - \Title},
    pdfauthor={\MyName},
    linkcolor=blue,
    citecolor=blue,
    filecolor=black,
    urlcolor=MidnightBlue
}



\begin{document}

% No header for the first page
\thispagestyle{empty}

%%%%%%%%%%%%%%%%%%%%%%%%%%%%%%%%%%%%%%%%%%%%%%%%%%%%%%%%%%%%%%%%%%%%%%%%%%%%%%%
% HEADER
\begin{tabular}{@{}l l@{}}
    \parbox{0.55\textwidth}{
        {\fontsize{36pt}{0}\selectfont \MyName}
        \\[0.5cm]
        {\fontsize{13pt}{0}\selectfont \Title \, | \, \monthyear\today}
    } &
    \parbox{0.45\textwidth}{
        \begin{flushright}
        \fontsize{10pt}{12pt}\selectfont
        Department of Geology and Geophysics, SOEST
        \\
        University of Hawai'i at Manoa
        \\
        1680 East-West Rd, POST 804
        \\
        96822 Honolulu, HI, USA
        \\
        ORCID: \href{http://orcid.org/\ORCID}{\ORCID}
        \\
        \href{mailto:\Email}{\Email}
        \, | \,
        \href{http://www.leouieda.com}{www.leouieda.com}
        \end{flushright}
    }
\end{tabular}

\vspace{0.5cm}


%%%%%%%%%%%%%%%%%%%%%%%%%%%%%%%%%%%%%%%%%%%%%%%%%%%%%%%%%%%%%%%%%%%%%%%%%%%%%%%
\section*{Professional experience}

\TablePad
\begin{tabularx}{\textwidth}{@{}l X}
    \Duration{02/2017}{present}  &
    \textbf{Visiting Research Scholar} (postdoc),
    University of Hawaii at Manoa, USA
    \newline
    \textit{Working with Prof. Paul Wessel to build a Python interface for
    the Generic Mapping Tools}
    \\
    \Duration{02/2014}{present}  &
    \textbf{Assistant Professor},
    Universidade do Estado do Rio de Janeiro, Brazil
    \newline
    \textit{Professor of Geophysics at the Department of Applied Geology.
    Responsible for the Laboratory of Exploration Geophysics (LAGEX).}
    \\
    \Duration{02/2011}{03/2011}  &
    \textbf{Visiting Researcher},
    University of Trieste, Italy
    \newline
    \textit{Working with professor Carla Braitenberg to develop version 1.0 of
    the open-source software Tesseroids.}
\end{tabularx}


%%%%%%%%%%%%%%%%%%%%%%%%%%%%%%%%%%%%%%%%%%%%%%%%%%%%%%%%%%%%%%%%%%%%%%%%%%%%%%%
\section*{Education}

\TablePad
\begin{tabularx}{\textwidth}{@{}l X}
    \Duration{11/2011}{04/2016}  &
    \textbf{PhD in Geophysics}, Observatório Nacional, Brazil
    \newline
    Thesis: \textit{Forward modeling and inversion of gravitational fields in
    spherical coordinates}
    %\newline
    \Code{https://github.com/leouieda/phd-thesis}
    \PDF{http://www.leouieda.com/about/phd.html}
    \\
    \Duration{03/2010}{10/2011}  &
    \textbf{MSc in Geophysics}, Observatório Nacional, Brazil
    \newline
    Thesis: \textit{Robust 3D gravity gradient inversion by planting anomalous
    densities}
    %\newline
    \Code{https://github.com/pinga-lab/paper-planting-densities}
    \PDF{http://www.leouieda.com/about/masters.html}
    \\
    \Duration{08/2008}{05/2009}  &
    \textbf{International Exchange}, York University, Canada
    \\
    \Duration{03/2004}{12/2009}  &
    \textbf{BSc in Geophysics}, Universidade de São Paulo, Brazil
    \newline
    Thesis: \textit{Forward modeling of the gravity gradient tensor using
    tesseroids}
    %\newline
    \Code{https://github.com/leouieda/barchelor-thesis}
    \PDF{http://www.leouieda.com/about/bachelors.html}
\end{tabularx}


%%%%%%%%%%%%%%%%%%%%%%%%%%%%%%%%%%%%%%%%%%%%%%%%%%%%%%%%%%%%%%%%%%%%%%%%%%%%%%%
\section*{Funding}

\TablePad
\begin{tabularx}{\textwidth}{@{}l X}
    \Duration{11/2014}{11/2017}  &
    QUALITEC/UERJ program for training a technician for the Laboratory of
    Exploration Geophysics
    \textbf{(BRL \$154,800 as a scholarship for the technician)}.
\end{tabularx}


%%%%%%%%%%%%%%%%%%%%%%%%%%%%%%%%%%%%%%%%%%%%%%%%%%%%%%%%%%%%%%%%%%%%%%%%%%%%%%%
\section*{Honors \& Awards}

\TablePad
\begin{tabularx}{\textwidth}{@{}p{0.15\textwidth} p{0.85\textwidth}}
    \Year{2017}  &
    Brazilian Geophysical Society (SBGf) Award for \textbf{Best PhD Thesis}
    of 2015-2017
    \\
    \Year{2016}  &
    Universidade do Estado do Rio de Janeiro, Brazil, School of Geology
    \textbf{Teaching Award} given by the graduating class of 2016
    \\
    \Duration{2011}{2015}  &
    Brazilian Ministry of Education CAPES \textbf{PhD Scholarship} (4 years)
    with Professor Valéria C. F. Barbosa
    \\
    \Year{2011}  &
    SEG Near Surface Geophysics Section \textbf{Student Travel Grant} to
    present at the SEG Annual Meeting, San Antornio, TX, USA
    \\
    \Year{2011}  &
    EAGE \textbf{PACE Student Travel Grant} to present at the 73rd EAGE
    Conference \& Exhibition, Vienna, Austria
    \\
    \Duration{2010}{2011}  &
    Brazilian Ministry of Education CAPES \textbf{Masters Scholarship} (2
    years) with Professor Valéria C. F. Barbosa
    \\
    \Year{2008}  &
    Brazilian Geophysical Society (SBGf) \textbf{Undergraduate Research
    Scholarship} (1 year) with Professor Naomi Ussami
    \\
    \Year{2005}  &
    São Paulo Research Foundation (FAPESP) \textbf{Undergraduate Research
    Scholarship} (1 year) with Professor Manoel S. D'Agrella Filho
\end{tabularx}


%%%%%%%%%%%%%%%%%%%%%%%%%%%%%%%%%%%%%%%%%%%%%%%%%%%%%%%%%%%%%%%%%%%%%%%%%%%%%%%
\section*{Open-source Software}

\TablePad
\begin{tabularx}{\textwidth}{@{}X X X@{}}
    \SoftwareTitle{Fatiando a Terra}
    \newline
    A Python library for geophysical data analysis, modeling, and
    inversion.
    \newline
    \href{https://github.com/fatiando/fatiando}{github.com/fatiando/fatiando}
    &
    \SoftwareTitle{Tesseroids}
    \newline
    Command-line programs for forward modeling of gravitational fields in
    spherical coordinates.
    \newline
    \href{https://github.com/leouieda/tesseroids}{github.com/leouieda/tesseroids}
    &
    \SoftwareTitle{GMT/Python}
    \newline
    A Python interface for the Generic Mapping Tools.
    \newline
    \href{https://github.com/GenericMappingTools/gmt-python}{github.com/GenericMappingTools/\newline gmt-python}
\end{tabularx}


%%%%%%%%%%%%%%%%%%%%%%%%%%%%%%%%%%%%%%%%%%%%%%%%%%%%%%%%%%%%%%%%%%%%%%%%%%%%%%%
\section*{Teaching}

\TablePad
\begin{tabularx}{\textwidth}{@{}X X@{}}
    \TableTitle{Undergraduate} & \TableTitle{Workshops/Short courses}
    \\[0.1cm]
    \begin{tabular}{@{}l l}
        \Duration{2014}{2016}  &
         matesp
        \hspace{10cm}
        \\
        \Duration{2014}{2016}  &
         Geofisica 1
        \\
        \Duration{2014}{2016}  &
         Geofisica 2
        \\
        \Year{2015}  &
         Intro Geology
    \end{tabular}
    &
    \begin{tabular}{@{}l l}
        \Year{2017}  &
        Short course Python UH
        \\
        \Year{2016}  &
        Python USP
        \\
        \Year{2014}  &
        Inversion UnB
        \\
        \Year{2011}  &
        Inversion USP
    \end{tabular}
\end{tabularx}


%%%%%%%%%%%%%%%%%%%%%%%%%%%%%%%%%%%%%%%%%%%%%%%%%%%%%%%%%%%%%%%%%%%%%%%%%%%%%%%
\section*{Publications}

\subsection*{Journal articles}

\TablePad
\begin{tabularx}{\textwidth}{@{}l X}
\Year{2017}  &
    \Me.
    Step-by-step NMO correction,
    \emph{The Leading Edge},
    \DOI{10.1190/tle36020179.1}.
    \PDF{http://www.leouieda.com/papers/nmo-tutorial.html}
    \Code{https://github.com/pinga-lab/nmo-tutorial}
    \\
    ~ &
    \Me, \Val.
    Fast non-linear gravity inversion in spherical coordinates with application
    to the South American Moho,
    \emph{Geophysical Journal International},
    \DOI{10.1093/gji/ggw390}.
    \PDF{http://www.leouieda.com/papers/paper-moho-inversion-tesseroids-2016.html}
    \Code{https://github.com/pinga-lab/paper-moho-inversion-tesseroids}
    \\
\Year{2016}  &
    \Me, \Val, \Carla.
    Tesseroids: forward modeling gravitational fields in spherical coordinates,
    \emph{Geophysics},
    \DOI{10.1190/geo2015-0204.1}.
    \PDF{http://www.leouieda.com/papers/paper-tesseroids-2016.html}
    \Code{https://github.com/pinga-lab/paper-tesseroids}
    \\
    ~ &
    \Dio, \Me, \Val.
    How two gravity-gradient inversion methods can be used to reveal different
    geologic features of ore deposit - A case study from the Quadrilátero
    Ferrífero (Brazil),
    \emph{Journal of Applied Geophysics},
    \DOI{10.1016/j.jappgeo.2016.04.011}.
    \PDF{http://www.leouieda.com/papers/paper-quadrilatero2-2016.html}
    \\
\Year{2015}  &
    \Bi, \Dai, \Val, \Me.
    Estimation of the total magnetization direction of approximately spherical
    bodies,
    \emph{Nonlinear Processes in Geophysics},
    \DOI{10.5194/npg-22-215-2015}.
    \PDF{http://www.leouieda.com/papers/paper-mag-dir-2015.html}
    \Code{https://github.com/pinga-lab/Total-magnetization-of-spherical-bodies}
    \\
\Year{2014}  &
    \Me, \Bi, \Val.
    Geophysical tutorial: Euler deconvolution of potential-field data,
    \emph{The Leading Edge},
    \DOI{10.1190/tle33040448.1}.
    \PDF{http://www.leouieda.com/papers/paper-tle-euler-tutorial-2014.html}
    \Code{https://github.com/pinga-lab/paper-tle-euler-tutorial}
    \\
    ~ &
    \Dio, \Me, \Val.
    Imaging iron ore from the Quadrilátero Ferrífero (Brazil) using geophysical
    inversion and drill hole data,
    \emph{Ore Geology Reviews},
    \DOI{10.1016/j.oregeorev.2014.02.011}.
    \PDF{http://www.leouieda.com/papers/paper-quadrilatero-2014.html}
    \\
\Year{2013}  &
    \Figura, \Val, \Me, \Bi, \JB.
    Estimating the nature and the horizontal and vertical positions of 3D
    magnetic sources using Euler deconvolution,
    \emph{Geophysics},
    \DOI{10.1190/geo2012-0515.1}.
    \PDF{http://www.leouieda.com/papers/paper-euler-plateau-2013.html}
    \\
    ~ &
    \Bi, \Val, \Me.
    Polynomial equivalent layer,
    \emph{Geophysics},
    \DOI{10.1190/geo2012-0196.1}.
    \PDF{http://www.leouieda.com/papers/paper-polynomial-eqlayer-2013.html}
    \\
\Year{2012}  &
    \Me, \Val.
    Robust 3D gravity gradient inversion by planting anomalous densities,
    \emph{Geophysics},
    \DOI{10.1190/geo2011-0388.1}.
    \PDF{http://www.leouieda.com/papers/paper-planting-anomalous-densities-2012.html}
    \Code{https://github.com/pinga-lab/paper-planting-densities}
\end{tabularx}


\subsection*{Conference proceedings}

\TablePad
\begin{tabularx}{\textwidth}{@{}l X}
\Year{2014}  &
    \Figura, \Val, \Me, \Bi, \JB.
    A Single Euler Solution Per Anomaly,
    \emph{76th EAGE Conference and Exhibition 2014},
    \DOI{10.3997/2214-4609.20140891}.
    \\
\Year{2013}  &
    \Me, \Bi, \Val.
    Modeling the Earth with Fatiando a Terra,
    \emph{Proceedings of the 12th Python in Science Conference}.
    \PDF{http://www.leouieda.com/talks/scipy2013.html}
    \Code{https://github.com/leouieda/scipy2013}
    \\
\Year{2012}  &
    \Me, \Val.
    Use of the ``shape-of-anomaly'' data misfit in 3D inversion by planting
    anomalous densities,
    \emph{SEG Technical Program Expanded Abstracts},
    \DOI{10.1190/segam2012-0383.1}.
    \PDF{http://www.leouieda.com/talks/seg2012.html}
    \Code{https://github.com/leouieda/seg2012}
    \\
    ~ &
    \Dio, \Me, \YLi, \Val, \BragaVale, \Angeli, \Peres.
    Iron ore interpretation using gravity-gradient inversions in the Carajás, Brazil.
    \emph{SEG Technical Program Expanded Abstracts},
    \DOI{10.1190/segam2012-0525.1}.
    \\
\Year{2011}  &
    \Me, \Everton, \Carla, \Eder.
    Optimal forward calculation method of the Marussi tensor due to a geologic
    structure at GOCE height,
    \emph{Proceedings of the 4th International GOCE User Workshop}.
    \PDF{http://www.leouieda.com/posters/goce2011.html}
    \Code{https://github.com/leouieda/goce2011}
    \\
    ~ &
    \Me, \Val.
    Robust 3D gravity gradient inversion by planting anomalous densities,
    \emph{SEG Technical Program Expanded Abstracts},
    \DOI{10.1190/1.3628201}.
    \PDF{http://www.leouieda.com/talks/seg2011.html}
    \Code{https://github.com/leouieda/seg2011}
    \\
    ~ &
    \Me, \Val.
    3D gravity inversion by planting anomalous densities.
    \emph{12th International Congress of the Brazilian Geophysical Society},
    \DOI{10.1190/sbgf2011-179}.
    \PDF{http://www.leouieda.com/talks/sbgf2011.html}
    \Code{https://github.com/leouieda/sbgf2011}
    \\
    ~ &
    \Me, \Val.
    3D gravity gradient inversion by planting density anomalies.
    \emph{73th EAGE Conference and Exhibition incorporating SPE EUROPEC},
    \DOI{10.3997/2214-4609.20149567}.
    \PDF{http://www.leouieda.com/posters/eage2011.html}
    \Code{https://github.com/leouieda/eage2011}
    \\
    ~ &
    \Dio, \Me, \Val, \BragaVale, \Gomes.
    In-depth imaging of an iron orebody from Quadrilatero Ferrifero using 3D
    gravity gradient inversion,
    \emph{SEG Technical Program Expanded Abstracts},
    \DOI{10.1190/1.3628219}.
    \\
    ~ &
    \Dio, \Val, \Me, \BragaVale.
    Inversão de Dados de Aerogradiometria Gravimétrica 3D-FTG Aplicada a
    Exploração Mineral na Região do Quadrilátero Ferrífero,
    \emph{12th International Congress of the Brazilian Geophysical Society},
    \DOI{10.1190/sbgf2011-243}.
\end{tabularx}


%%%%%%%%%%%%%%%%%%%%%%%%%%%%%%%%%%%%%%%%%%%%%%%%%%%%%%%%%%%%%%%%%%%%%%%%%%%%%%%
\section*{Presentations
          \lowercase{\fontsize{11pt}{0}\selectfont (presenting author only)}}

\TablePad
\begin{tabularx}{\textwidth}{@{}l X}
\Year{2017}  &
    \Me, et al.
    Bringing the Generic Mapping Tools to Python,
    \emph{Scipy 2017},
    Austin, USA.
    \Youtube{93M4How7R24}
    \Slides{http://www.leouieda.com/talks/scipy2017.html}
    \\
    ~ &
    \Me.
    Inverting gravity to map the Moho: A new method and the open source
    software that made it possible,
    \emph{University of Hawaii},
    Honolulu, USA.
    \Invited{}
    \Slides{http://www.leouieda.com/talks/tgif-2017.html}
    \\
\Year{2015}  &
    \Me.
    Fatiando a Terra: construindo uma base para ensino e pesquisa de geofísica,
    \emph{Universidade de São Paulo},
    São Paulo, Brazil.
    \Invited{}
    \Slides{http://www.leouieda.com/talks/iag-04-2015.html}
    \\
\Year{2014}  &
    \Me, et al.
    Using Fatiando a Terra to solve inverse problems in geophysics,
    \emph{Scipy 2014},
    Austin, USA.
    \Poster{http://www.leouieda.com/posters/scipy2014.html}
    \\
    ~ &
    \Me, et al.
    Gravity inversion in spherical coordinates using tesseroids,
    \emph{EGU General Assembly 2014},
    Vienna, Austria.
    \Slides{http://www.leouieda.com/talks/egu2014.html}
    \\
\Year{2013}  &
    \Me, et al.
    Modeling the Earth with Fatiando a Terra,
    \emph{Scipy 2013},
    Austin, USA.
    \Youtube{Ec38h1oB8cc}
    \Slides{http://www.leouieda.com/talks/scipy2013.html}
    \\
    ~ &
    \Me, et al.
    3D magnetic inversion by planting anomalous densities,
    \emph{AGU Meeting of the Americas},
    Cancun, Mexico.
    \Slides{http://www.leouieda.com/talks/agu-cancun2013.html}
    \\
\Year{2012}  &
    \Dio, \Me, et al.
    Iron ore interpretation using gravity-gradient inversions in the Carajás,
    Brazil,
    \emph{SEG Annual Meeting 2012},
    Las Vegas, USA.
    \Slides{http://www.leouieda.com/talks/seg-carlos2012.html}
    \\
    ~ &
    \Me, et al.
    Use of the ``shape-of-anomaly'' data misfit in 3D inversion by planting
    anomalous densities,
    \emph{SEG Annual Meeting 2012},
    Las Vegas, USA.
    \Slides{http://www.leouieda.com/talks/seg2012.html}
    \\
    ~ &
    \Me, et al.
    Rapid 3D inversion of gravity and gravity gradient data to test geologic
    hypotheses,
    \emph{International Symposium on Gravity, Geoid and Height Systems},
    Venice, Italy.
    \Slides{http://www.leouieda.com/talks/gghs2012.html}
    \\
\Year{2011}  &
    \Me, et al.
    Robust 3D gravity gradient inversion by planting anomalous densities,
    \emph{SEG Annual Meeting 2011},
    San Antonio, USA.
    \Slides{http://www.leouieda.com/talks/seg2011.html}
    \\
    ~ &
    \Me, et al.
    3D gravity inversion by planting anomalous densities,
    \emph{Internation Congress of the Brazilian Geophysical Society},
    Rio de Janeiro, Brazil.
    \Slides{http://www.leouieda.com/talks/sbgf2011.html}
    \\
    ~ &
    \Me, et al.
    Optimal forward calculation method of the Marussi tensor due to a geologic
    structure at GOCE height,
    \emph{4th International GOCE User Workshop},
    Munich, Germany.
    \Poster{http://www.leouieda.com/posters/goce2011.html}
    \\
    ~ &
    \Me, et al.
    3D gravity gradient inversion by planting density anomalies,
    \emph{73th EAGE Conference and Exhibition incorporating SPE EUROPEC},
    Vienna, Austria.
    \Poster{http://www.leouieda.com/posters/eage2011.html}
    \\
\Year{2010}  &
    \Me, et al.
    Computation of the gravity gradient tensor due to topographic masses using
    tesseroids,
    \emph{AGU Meeting of the Americas},
    Foz do Iguaçu, Brazil.
    \Slides{http://www.leouieda.com/talks/agu2010.html}
    \\
\Year{2008}  &
    \Me, et al.
    Utilização de tesseróides na modelagem de dados de gradiometria
    gravimétrica,
    \emph{XIII Simpósio de Iniciação Científica do IAG-USP},
    São Paulo, Brazil.
    \Poster{http://www.leouieda.com/posters/simposio-iag-2008.html}
    \\
\Year{2006}  &
    \Me, et al.
    Paleomagnetismo e mineralogia magnética dos diques cambrianos de Maravilhas
    e Prata (PB),
    \emph{XI Simpósio de Iniciação Científica do IAG/USP},
    São Paulo, Brazil.
    \Poster{http://www.leouieda.com/posters/simposio-iag-2006.html}
\end{tabularx}


%%%%%%%%%%%%%%%%%%%%%%%%%%%%%%%%%%%%%%%%%%%%%%%%%%%%%%%%%%%%%%%%%%%%%%%%%%%%%%%
\section*{Skills}

\TablePad
\begin{tabularx}{\textwidth}[t]{@{}p{0.5\textwidth} p{0.5\textwidth}@{}}
    \TableTitle{Languages} &
    \TableTitle{Programming}
    \\[0.1cm]
    \begin{tabular}[t]{@{}l}
        \Item English: fluent (TOEFL iBT score 115/120)
        \\
        \Item Portuguese: native
        \\
        \Item Spanish: basic
    \end{tabular}
    &
    \begin{tabular}[t]{@{}l}
        \Item Python (main language since 2008)
        \\
        \Item C
        \\
        \Item Bash
        \\
        \Item Version control (git, mercurial, subversion)
        \\
        \Item HTML \& CSS
        \\
        \Item LaTeX
    \end{tabular}
\end{tabularx}


%%%%%%%%%%%%%%%%%%%%%%%%%%%%%%%%%%%%%%%%%%%%%%%%%%%%%%%%%%%%%%%%%%%%%%%%%%%%%%%
\section*{Service}

\TablePad
\begin{tabularx}{\textwidth}[t]{@{}p{0.5\textwidth} p{0.5\textwidth}@{}}
    \TableTitle{Affiliations}
    &
    \TableTitle{Reviewer}
    \\[0.1cm]
    \begin{tabular}[t]{@{}l}
        \Item American Geophysical Union
        \\
        \Item Society of Exploration Geophysicists
        \\
        \Item Geological Society of America
    \end{tabular}
    &
    \begin{tabular}[t]{@{}l}
        \Item Computers \& Geosciences
        \\
        \Item Geophysics
        \\
        \Item Central European Journal of Geosciences
        \\
        \Item Pure and Applied Geophysics
        \\
        \Item Journal of Applied Geophysics
        \\
        \Item Geophysical Prospecting
        \\
        \Item Geophysical Journal International
        \\
        \Item Journal of Geodesy
    \end{tabular}
\end{tabularx}

\end{document}
