\documentclass[12pt]{moderncv}
\moderncvtheme[grey]{classic}
\usepackage[left=1in,right=1in,top=1in,bottom=1in]{geometry}

\usepackage{graphicx}
\usepackage{url}
\usepackage[utf8]{inputenc}

\definecolor{color1}{rgb}{0.08,0.27,0.47} % Headers
%\definecolor{color1}{rgb}{0.25,0.25,0.25} % Headers
%\definecolor{color1}{rgb}{0.30,0.30,0.30} % Headers
\definecolor{color4}{rgb}{0.45,0.45,0.45} % Title and information

\usepackage{fancyhdr}
\pagestyle{fancy}
\fancyhf{}
\chead{Leonardo Uieda}
\rhead{\thepage}

\firstname{Leonardo}
\familyname{Uieda}
\title{\emph{Curriculum Vit\ae\ - 06/2016}}
\address{
    Universidade do Estado do Rio de Janeiro
    }{
    R. São Francisco Xavier, 524/2019A
    }{
    20550-900 Rio de Janeiro - Brazil
}{}
\extrainfo{
    +55~21~983636761
    \\
    \href{mailto:leouieda@gmail.com}{\emph{\texttt{leouieda@gmail.com}}}
    \\
    \href{http://www.leouieda.com}{\emph{\texttt{www.leouieda.com}}}
}


\begin{document}

\thispagestyle{empty}
\maketitle

\vspace{-1.5cm}
\hrulefill
\vspace{0.5cm}


\section{Research interests}

\begin{cvcolumns}
  \cvcolumn{}{\begin{itemize}
    \item Open-source scientific software
    \item Gravity and magnetics
    \item Reproducibility of computations
  \end{itemize}}
  \cvcolumn{}{\begin{itemize}
    \item Inverse problem theory
    \item Large-scale gravity interpretation
    \item Numerical modeling
  \end{itemize}}
\end{cvcolumns}


\section{Education}

\cventry{2011--2016}{PhD in Geophysics}{}{Observat\'orio Nacional}{Rio de Janeiro, Brazil}{Thesis:
    Forward modeling and inversion of gravitational fields in spherical
    coordinates\\
    Advisor: Valéria C. F. Barbosa\\
    Source-code and PDF:
    \href{http://www.leouieda.com/about/phd.html}{\texttt{leouieda.com/about/phd.html}}}

\cventry{2010--2011}{MSc in Geophysics}{}{Observat\'orio Nacional}{Rio de Janeiro, Brazil}{Dissertation:
    3D gravity gradient inversion by planting anomalous densities\\
    Advisor: Valéria C. F. Barbosa\\
    Source-code and PDF:
    \href{http://www.leouieda.com/about/masters.html}{\texttt{leouieda.com/about/masters.html}}}

\cventry{2008--2009}{International Exchange}{}{York University}{Toronto, Canada}{}

\cventry{2004--2009}{BSc in Geophysics}{}{Universidade de S\~ao Paulo}{S\~ao Paulo, Brazil}{Dissertation:
    Use of tesseroids in the modeling of gravity gradiometry data\\
    Advisor: Naomi Ussami\\
    Source-code and PDF:
    \href{http://www.leouieda.com/about/bachelors.html}{\texttt{leouieda.com/about/bachelors.html}}}


\section{Professional experience}

\cventry{2014--present}{Assistant Professor}{}{Universidade do Estado do Rio de
    Janeiro}{Brazil}{Professor of Geophysics at the Department of Applied
    Geology. Responsible for the Laboratory of Exploration Geophysics (LAGEX).
    My position is the Brazilian equivalent of a tenure-track position.}

\cventry{Feb 2011}{Visiting researcher}{}{University of Trieste}{Trieste, Italy}{
    Working with professor Carla Braitenberg to develop version 1.0 of my
    open-source software Tesseroids.}


\section{Grants}

\cvitem{2014--2017}{QUALITEC/UERJ program for training a technician for the
    Laboratory of Exploration Geophysics (LAGEX).
    Total BRL \$154,800 as a scholarship for the technician.}


\section{Software}

I am the creator and main developer of the following open-source software:
\\

\cventry{\raisebox{-1.7cm}{\includegraphics[width=1.8cm]{figures/fatiando.png}}
    }{Fatiando a Terra
    {\normalfont (\href{http://www.fatiando.org}{\texttt{www.fatiando.org})}}}{}{}{}{
    \emph{A Python library for modeling and inversion in geophysics.}\\
    Provides tools for geophysical data analysis (mainly potential fields),
    forward modeling, and prototyping and implementing new inversion methods.\\
    Source-code:
    \href{https://github.com/fatiando/fatiando}{\texttt{github.com/fatiando/fatiando}}
}

\vspace{0.3cm}
\cventry{\raisebox{-1.7cm}{\includegraphics[height=1.7cm]{figures/tesseroids.png}}
    }{Tesseroids
    {\normalfont (\href{http://tesseroids.leouieda.com}{\texttt{tesseroids.leouieda.com})}}}{}{}{}{
    \emph{Forward modeling of gravitational fields in spherical coordinates.}\\
    Command-line programs (written in C) for modeling the gravitational potential,
    acceleration, and gradient tensor in Cartesian and spherical coordinates.\\
    Source-code:
    \href{https://github.com/leouieda/tesseroids}{\texttt{github.com/leouieda/tesseroids}}
}


\section{Publications}

Peer-reviewed articles (including links to PDFs and source-code repositories
hosted on \href{https://www.github.com}{\texttt{github.com}}):
\\

\cvitem{Submitted}{\textbf{Uieda, L} and VCF Barbosa, Fast non-linear
    gravity inversion in spherical coordinates with application to the South
    American Moho.
    \texttt{[\href{http://www.leouieda.com/papers/paper-moho-inversion-tesseroids-2016.html}{pdf}]}\texttt{[\href{https://github.com/pinga-lab/paper-moho-inversion-tesseroids}{code}]}
}

\cvitem{2016}{Carlos, DU, \textbf{L Uieda}, and VCF Barbosa, How two
    gravity-gradient inversion methods can be used to reveal different geologic
    features of ore deposit - A case study from the Quadrilátero Ferrífero
    (Brazil), \emph{Journal of Applied Geophysics},
    doi:10.1016/j.jappgeo.2016.04.011.
\texttt{[\href{http://www.leouieda.com/papers/paper-quadrilatero2-2016.html}{pdf}]}}

\vspace{0.3cm}
Peer-reviewed full conference papers:
\\

\cvitem{2016}{Carlos, D. U., L. Uieda, and V. C. F. Barbosa, How two
gravity-gradient inversion methods can be used to reveal different geologic
features of ore deposit - A case study from the Quadrilátero Ferrífero
(Brazil), \emph{Journal of Applied Geophysics},
doi:10.1016/j.jappgeo.2016.04.011. \texttt{[\href{http://www.leouieda.com/papers/paper-quadrilatero2-2016.html}{pdf}]}}

\vspace{0.3cm}
Other:
\\

\cvitem{2016}{Carlos, D. U., L. Uieda, and V. C. F. Barbosa, How two
gravity-gradient inversion methods can be used to reveal different geologic
features of ore deposit - A case study from the Quadrilátero Ferrífero
(Brazil), \emph{Journal of Applied Geophysics},
doi:10.1016/j.jappgeo.2016.04.011. \texttt{[\href{http://www.leouieda.com/papers/paper-quadrilatero2-2016.html}{pdf}]}}

\section{Invited talks}

\section{Teaching}

\section{Student supervision}

\section{Professional activities}

Reviewer for journals etc.

\section{Languages}

\cvitem{Portuguese}{Native}
\cvitem{English}{Fluent (TOEFL iBT score 115 out of 120 obtained 14 Sep 2007)}

\section{References}


\end{document}
