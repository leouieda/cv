\documentclass[12pt]{article}
\usepackage[left=1in,right=1in,top=0.1in,bottom=1in]{geometry}

\usepackage{graphicx}
\usepackage{url}
\usepackage[utf8]{inputenc}

\title{\textbf{Statement of Research Objectives}}
\author{Leonardo Uieda}
\date{June 2016}

\begin{document}

\maketitle

% Why is my research important?
% How will I approach it?
% What are my long term research goals?
% What are my career goals?

As a geophysicist,
my ultimate goal is
to infer knowledge about the inner Earth
and its processes
from surface observations,
such as its gravity and magnetic fields,
topography,
or propagation patterns of seismic waves.
%
Ubiquitous to all of geophysics
is the fact that this inference
is an ill-posed inverse problem,
to which a solution might not exist
or be non-unique and unstable.


My main research focus is
the development of methods
to solve inverse problems,
mainly for potential fields.
%
Central to all of my projects
is the open-source software
that implements the new methods.


\textbf{Software}:
Methodological development requires
much prototyping and iteration.
%
Thus,
a researcher needs
a flexible environment
and a large collection of tools
for experimentation.
%
The approach I have taken is
to develop an open-source library
called Fatiando a Terra\footnote{\url{http://www.fatiando.org}}
that collects the basic tools
required for building an inversion method.
%
The library is implemented in Python,
a dynamically typed interpreted language
known for its simplicity
and large ecosystem of scientific libraries.
%
Fatiando is developed in the
open\footnote{\url{https://github.com/fatiando/fatiando}}
with the help of a growing, though yet small,
developer community.
%
I use it as the basis for
all of my research projects
as well as for teaching geophysics.
%
My first open-source project
was \textit{Tesseroids}\footnote{\url{http://tesseroids.leouieda.com/}},
a collection of command-line programs
for forward modeling gravitational fields
using spherical prisms.
%
It is written entirely in C
and is my most widely used software project to date.


\textbf{Inverse problems}:
During my graduate studies,
I have developed two novel inversion methods
for gravity data.
%
The first is a 3D gravity gradient inversion
using a heuristic algorithm
that grows the solution from starting ``seeds''.
%
The method is computationally efficient
and can handle problems with millions of unknowns
on a mid-range laptop computer.
%
The second method inverts gravity data
to estimate the relief
of the crust-mantle boundary
using a spherical approximation for the Earth.
%
A common theme of my research has been
to use, adapt, and improve upon
highly efficient algorithms
to solve the problems of geophysical inversion.


\textbf{Reproducibility}:
Computational experiments
are difficult, if not impossible, to reproduce
without access to the code used to generate them.
%
I attempt to tackle
some of these issues
on my own research
by making all source-code and data
(as much as legally possible)
from my own publications
available in public repositories\footnote{For example \url{https://github.com/pinga-lab/paper-moho-inversion-tesseroids}}.
%
I hope to train my students
in these practices from the start
and provide guidance for others to do the same.
%
Though I cannot claim
to generate fully reproducible results,
I have been refining this process over time
and hope that my efforts will contribute
to advance the reproducibility of our science.

\end{document}
