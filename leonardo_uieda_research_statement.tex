\documentclass[12pt]{article}
\usepackage[left=1in,right=1in,top=1in,bottom=1in]{geometry}

\usepackage{graphicx}
\usepackage{url}
\usepackage[utf8]{inputenc}

\title{\textbf{Statement of Research Objectives}}
\author{Leonardo Uieda}
\date{June 2016}

\begin{document}

\maketitle


\textbf{Why is my research important?
How will I approach it?
What are my long term research goals?
What are my career goals?}

As a geophysicist,
my ultimate goal is
to infer knowledge about the inner Earth
and its processes
from surface observations,
such as it's gravity and magnetic fields,
topography,
or propagation patters of seismic waves.
%
Ubiquitous to all geophysical methods is
the fact that this inference is
an ill-posed inverse problem,
to which a solution might not exist or be non-unique and unstable.
%
Cutting edge research
in geophysical inversion
requires dominion not only of the mathematical basis
but of the computational tools currently available.
%
My main research focus is
the development of new inversion methods
and the software that implements them.
%
The approach I have taken is
to develop an open-source Python library
called Fatiando a Terra\footnote{\url{http://www.fatiando.org}}
(Portuguese for Slicing the Earth)
that collects the basic tools
for building a geophysical inversion method.
%
The library is developed in the open
with the help of a growing, though yet small,
developer community.
%
The use of freely available open-source technology
allows me to attempt to tackle
the problem of the reproducibility of my computational results.
%
I am doing this by
making all source-code and data (as much as legally possible)
from my own publications
available in public
repositories\footnote{For example
\url{https://github.com/pinga-lab/paper-moho-inversion-tesseroids}}.
%
I hope to train my own students
in these practices from the start
and provide guidance for others to do the same.
%
Though I cannot claim
to generate fully reproducible results,
I have been refining this process over time
and hope that my efforts will contribute
to advance the reproducibility of science.


Creating a new inversion method
requires a solid base of well tested
forward modeling and linear algebra code.
%




Projects and what I have done.

Future research.

\end{document}
