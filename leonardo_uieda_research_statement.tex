\documentclass[12pt]{article}
\usepackage[left=1in,right=1in,top=0.1in,bottom=1in]{geometry}

\usepackage{graphicx}
\usepackage{url}
\usepackage[utf8]{inputenc}

\title{\textbf{Statement of Research Objectives}}
\author{Leonardo Uieda}
\date{June 2016}

\begin{document}

\maketitle

% Why is my research important?
% How will I approach it?
% What are my long term research goals?
% What are my career goals?

As a geophysicist,
my ultimate goal is
to infer knowledge about the inner Earth
and its processes
from surface observations,
such as its gravity and magnetic fields,
topography,
or propagation patterns of seismic waves.
%
Ubiquitous to all of geophysics
is the fact that this inference
is an ill-posed inverse problem,
to which a solution might not exist
or be non-unique and unstable.


My main research focus is
the development of methods
to solve inverse problems,
mainly for potential fields.
%
Innovation can come from
a combination of factors:
different parametrizations,
more robust algorithms,
new forms of regularization,
etc.
%
Central to all of my projects
is the open-source software
that implements the new methods.


\textbf{Software}:
Methodological development requires
much prototyping and iteration
to arrive at a final product.
%
Thus,
the researcher needs
a flexible environment
and a large collection of tools
for experimentation.
%
The approach I have taken is
to develop an open-source library
called Fatiando a Terra\footnote{\url{http://www.fatiando.org}}
(Portuguese for Slicing the Earth)
that collects the basic tools
required for building an inversion method.
%
The library is implemented in Python,
a dynamically typed interpreted language
known for its simplicity
and large ecosystem of scientific libraries.



\textbf{Inversion methods}:




%\textbf{Gravity interpretation}:
\textbf{Reproducibility}:



%
The library is developed in the
open\footnote{\url{https://github.com/fatiando/fatiando}}
with the help of a growing, though yet small,
developer community.
%
The use of freely available open-source technology
allows me to attempt to tackle
the problem of the reproducibility of my computational results.
%
I am doing this by
making all source-code and data (as much as legally possible)
from my own publications
available in public
repositories\footnote{For example
\url{https://github.com/pinga-lab/paper-moho-inversion-tesseroids}}.
%
I hope to train my own students
in these practices from the start
and provide guidance for others to do the same.
%
Though I cannot claim
to generate fully reproducible results,
I have been refining this process over time
and hope that my efforts will contribute
to advance the reproducibility of our science.


% Projects and what I have done.

My research into gravity modeling
started during my undergraduate thesis.
%
The project was to develop an open-source application
for the forward modeling of the gravity gradient tensor
using tesseroids (spherical prisms).
%
The software would be used
for modeling data from the GOCE satellite,
which had not yet been launched at the time.
%
This was the starting point
of what later became
my most widely used software project,
\textit{Tesseroids}\footnote{\url{http://tesseroids.leouieda.com/}}.
%
During my Master's degree,
I developed a new 3D inversion method
for gravity gradient data.
I started development of Fatiando a Terra


% Future research.

\end{document}
