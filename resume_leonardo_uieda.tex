\documentclass[11pt,a4paper,helvetica]{moderncv}
\moderncvtheme[blue]{classic}
%\nopagenumbers{}
\usepackage[scale=0.85,top=1cm,bottom=1cm]{geometry}
\usepackage{natbib}

%\usepackage[brazilian]{babel}
%\usepackage[ansinew]{inputenc}
%\usepackage[T1]{fontenc}
%\usepackage{hyperref}
%\hypersetup{colorlinks=true}
%\hypersetup{urlcolor=blue}


\firstname{Leonardo}
\familyname{Uieda}
\photo[64pt][0.4pt]{picture}
\title{Geophysicist}
\address{Rua General Jose Cristino, 77}{20921-400 Rio de Janeiro - RJ - Brazil}{}
\email{leouieda@gmail.com}
\homepage{www.fatiando.org}
\extrainfo{\\[-0.3cm]
\includegraphics[height=10pt]{twitter}~\href{https://twitter.com/leouieda}{@leouieda} |
\includegraphics[height=10pt]{linkedin}~\href{http://www.linkedin.com/in/uieda}{linkedin.com/in/uieda} \\
\includegraphics[height=10pt]{github}~\href{https://github.com/leouieda}{github.com/leouieda} |
\includegraphics[height=10pt]{figshare}~\href{http://figshare.com/authors/Leonardo Uieda/97471}{figshare.com}
}

\begin{document}

\maketitle

\vspace{-1cm}

\section{Research interests}
\cvlistdoubleitem{Inverse problems}{Potential fields}
\cvlistdoubleitem{Open-source software}{Numerical modeling}

\section{Education}
\cventry{2011--Present}{PhD in Geophysics}{Observat\'orio Nacional}{Rio de Janeiro}{}{Thesis topic:
    \textit{Potential field inversion in spherical coordinates}}
\cventry{2010--2011}{MSc in Geophysics}{Observat\'orio Nacional}{Rio de Janeiro}{}{Dissertation topic:
    \textit{3D gravity gradient inversion by planting anomalous densities}}
\cventry{2008--2009}{International Exchange}{York University}{Toronto, Canada}{}{}
\cventry{2004--2009}{BSc in Geophysics}{Universidade de S\~ao Paulo}{S\~ao Paulo}{}{Dissertation topic:
        \textit{Use of tesseroids in the modeling of gravity gradiometry data}}

\section{Open-source software}
\cventry{\raisebox{-1.5cm}{\includegraphics[height=1.5cm]{fatiando}}}{Fatiando a Terra}{}{}{}{
    Python toolkit for geophysical modeling and inversion\\
    Website: \url{http://fatiando.org} \\
    Language: Python \\
    License: BSD}
\vspace{0.3cm}
\cventry{\raisebox{-1.5cm}{\includegraphics[height=1.5cm]{tesseroids.png}}}{Tesseroids}{}{}{}{
    Forward modeling of gravitational fields in spherical coordinates\\
    Website: \url{http://leouieda.github.com/tesseroids}\\
    Language: C \\
    License: BSD}

\section{Languages}
\cvitemwithcomment{Fluent}{Portuguese}{Native}
\cvitemwithcomment{Fluent}{English}{TOEFL score: 115/120 (received 10/2007)}
\cvitemwithcomment{Basic}{Spanish}{}

%\nocite{*}
%\bibliographystyle{chicago}
%\bibliography{publications}

\begin{thebibliography}{}
%\itemsep0pt

\bibitem[Uieda, 2013]{meh} {\small See
\url{http://fatiando.org/people/uieda} for a full list and access to PDFs.}

\bibitem[Uieda and Barbosa, 2012a]{Uieda2012a}
Uieda, L., and V.~C.~F. Barbosa,  2012, {Robust 3D gravity gradient inversion
  by planting anomalous densities}: Geophysics, 77, G55--G66,
  doi:10.1190/geo2011-0388.1.

\bibitem[{Oliveira Jr.} et~al., 2012]{OliveiraJr.2012}
{Oliveira Jr.}, V.~C., V.~C.~F. Barbosa, and L. Uieda,  2012, {Polynomial
  equivalent layer}: Geophysics, 78, G1--G13, doi:10.1190/geo2012-0196.1.

\bibitem[Uieda and Barbosa, 2012b]{Uieda2012d}
Uieda, L., and V.~C.~F. Barbosa, 2012, {Use of the ``shape-of-anomaly'' data
  misfit in 3D inversion by
  planting anomalous densities}: SEG Technical Program Expanded Abstracts,
  1--6, doi:10.1190/segam2012-0383.1.

\bibitem[Uieda et~al., 2011]{Uieda2011b}
Uieda, L., E.~P. Bomfim, C. Braitenberg, and E. Molina,  2011, {Optimal forward
  calculation method of the Marussi tensor due to a geologic structure at GOCE
  height}: Proc. of ``4th International GOCE User Workshop'', 1--5.

\end{thebibliography}
\end{document}
