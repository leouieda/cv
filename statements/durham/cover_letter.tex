\documentclass[11pt]{letter}
\usepackage[utf8]{inputenc}
\usepackage{geometry}
\usepackage{graphicx}
\usepackage{fancyhdr}
% Control the font size
\usepackage{anyfontsize}

% Set the color of hyperlinks
\usepackage[colorlinks=true]{hyperref}
\hypersetup{
    pdftitle={Cover Letter},
    pdfauthor={Leonardo Uieda},
    linkcolor=black,
    citecolor=black,
    filecolor=black,
    urlcolor=black
}

\longindentation=0pt

% Redefine the empty style so the front page also has a header.
\fancypagestyle{empty}{ %
    \fancyhf{} % remove everything
    \lhead{}
    %\cfoot{\thepage}
    \cfoot{}
    \renewcommand{\headrulewidth}{0pt}
    \renewcommand{\footrulewidth}{0pt}
    \setlength\headheight{31pt}
}
\pagestyle{empty}

\signature{Leonardo Uieda, PhD}
\address{
    1680 East-West Road, POST 821
    \\
    Honolulu, HI, USA, 96822
    \\
    email: \href{mailto:leouieda@gmail.com}{leouieda@gmail.com}
    \\
    phone: +1 808 428 4521
}

\begin{document}

\begin{letter}{
        Department of Earth Sciences,  \\
        Durham University, \\
        Science Labs, \\
        Durham DH1 3LE
}
\opening{Dear Members of the Search Committee:}

I am writing to apply for the position of
Assistant Professor in Earth Sciences
at the Department of Earth Sciences of Durham University.
I am currently a Visiting Research Scholar at the University of Hawaii at Manoa in the
Department of Earth Sciences.
My background is in geophysics with a focus on gravity and magnetic inverse problems and
scientific computing.

Before coming to Hawaii, I worked for three years as Assistant Professor at the School
of Geology of the State University of Rio de Janeiro (UERJ), Brazil, where I gained
valuable experience in student supervision, teaching at the undergraduate level, and
curriculum development.
I supervised an undergraduate student on a research project and I am co-advising a PhD
student (Santiago Soler) from the Universidad Nacional de San Juan, Argentina, who
recently submitted his first paper to the \textit{Geophysical Journal International}.
At uERJ, I had the opportunity to design and implement three new undergraduate classes:
two geophysics classes for the Geology program, one on gravity and magnetics and another
on exploration seismology, and a programming and numerical methods class for the
Oceanography program.
My strategy for developing these classes was based on current research in active
learning, which has been shown to lead to higher retention and deeper understanding of
the subject matter.
The students used their time in class to work on practical exercises and projects,
most of which were designed using interactive computational tools that I developed using
Python and Jupyter notebooks (a web-based literate programming tool with support for
interactive graphs and animations).
For example, the seismology modules were composed of Jupyter notebooks which ran and
animated simulations of seismic wave propagation, seismic acquisitions, and data
processing.
The programming modules were entirely implemented using the Github website and their
Github Classroom platform.
The students submit their work as Github repositories and receive feedback through the
platform, which allowed me to manage a project-based class with over 70 students.
This hands-on approach was well received by the students, as evidenced by high
grades and attendance rates and a teaching award given to me by the graduating class of
2016 of the geology program.

With my background in geophysics and my experiences at UERJ, I feel confident in my
ability to teach at a high level and outside of my immediate expertise, including
mathematics, geoinformatics, and introductory modules across the department's courses.


Admin
* Evidence of p articipation in the collegial/administrative
* Engagement in activities that contribute to the administrative functioning of an academic Department, Faculty, University and/or discipline including leadership or responsibilities in an academic context
* Candidates must have excellent oral and written communication skills with the ability to engage with a range of students and co lle a g u es across a variety of forums

In 2018, I left my post at UERJ because of the aggravated economic crisis and increase
in violence that made life in Rio de Janeiro unbearable.

My preparations for the seismic processing modules lead to the publication of an
open-access tutorial on the \textit{The Leading Edge} about the normal moveout
correction.


Evidence of my public speaking skills is also publicly available through my
talks at the Scipy Conference, which have been recorded and uploaded to YouTube.




Outline admin work at UERJ and UH.


Points from the ad:

Reasearch:
* outputs
* plan enhances department plans
* Research leadership
* PhD supervision
* Research impact beyond my institution
* income generation (participation in successfull research project proposals)







% Research
My research training was in the development of computationally efficient
algorithms for solving geophysical inverse problems.
The focus of my graduate research was in estimating subsurface density
variations from observation of disturbances in the Earth's gravity field (e.g.,
Uieda and Barbosa, 2012, 2017).
Recently, I have been investigating the application of machine learning and
data science techniques to geophysical datasets,
in particular to long-term GPS measurements
(Uieda and Wessel, 2018;
\href{https://github.com/leouieda/aogs2018-gps}{github.com/leouieda/aogs2018-gps}).
I intend to continue my research program in the intersection of geophysics and
data science and would welcome collaborations with the Data Science Lab.
I would also use this opportunity to establish a relationship the members of
the Department of Earth, Ocean and Atmospheric Sciences and the
UBC Geophysical Inversion Facility.

% Software
To support my research and teaching efforts, I have created the open-source
Python library \textit{Fatiando a Terra}
(\href{http://www.fatiando.org/}{www.fatiando.org}).
The project uses software development best practices, such as version control
and the Github pull request workflow,
automated tests and continuous integration,
and packaging and distribution through the Python Package Index and
conda-forge.
My current work at the University of Hawai'i is to develop a Python wrapper
library for the \textit{Generic Mapping Tools}
(GMT; \href{http://gmt.soest.hawaii.edu/}{gmt.soest.hawaii.edu}),
an open-source software packaged widely used across the Earth, Atmospheric, and
Ocean Sciences to process and visualize geospatial data.
I have also contributed to other open-source projects, all of which can be
accessed through my Github profile
(\href{https://github.com/leouieda/}{github.com/leouieda}).

% Open science
As a proponent of open and reproducible science, I publish all of the
source-code and data for my first author publications through my research
group's Github page
(\href{https://github.com/pinga-lab}{github.com/ pinga-lab}).
To promote reproducible research best practices within the group,
I maintain a template
(\href{https://github.com/pinga-lab/paper-template}{github.com/pinga-lab/paper-template})
which is used for creating new research projects.
I am interested in exchanging information and learning from the experiences of
the Data Science Workflows course of the MDS program.


% Conclusion
I look forward to the opportunity to learn from the experience of the MDS
faculty and to use my geoscience expertise to expand and enrich the program.
I am particularly interested in the Capstone projects, for which I see great
potential for collaboration with the mining and oil and gas industries.

\closing{Thank you for your consideration,}


\end{letter}

\end{document}
