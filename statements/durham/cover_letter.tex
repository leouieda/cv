\documentclass[11pt]{letter}
\usepackage[utf8]{inputenc}
\usepackage{geometry}
\usepackage{graphicx}
\usepackage{fancyhdr}
% Control the font size
\usepackage{anyfontsize}

% Set the color of hyperlinks
\usepackage[colorlinks=true]{hyperref}
\hypersetup{
    pdftitle={Cover Letter},
    pdfauthor={Leonardo Uieda},
    linkcolor=black,
    citecolor=black,
    filecolor=black,
    urlcolor=black
}

\longindentation=0pt

% Redefine the empty style so the front page also has a header.
\fancypagestyle{empty}{ %
    \fancyhf{} % remove everything
    \lhead{}
    %\cfoot{\thepage}
    \cfoot{}
    \renewcommand{\headrulewidth}{0pt}
    \renewcommand{\footrulewidth}{0pt}
    \setlength\headheight{31pt}
}
\pagestyle{empty}

\signature{Leonardo Uieda, PhD}
\address{
    1680 East-West Road, POST 821
    \\
    Honolulu, HI, USA, 96822
    \\
    email: \href{mailto:leouieda@gmail.com}{leouieda@gmail.com}
    \\
    phone: +1 808 428 4521
}

\begin{document}

\begin{letter}{
        Department of Earth Sciences,  \\
        Durham University, \\
        Science Labs, \\
        Durham DH1 3LE
}
\opening{Dear Members of the Search Committee:}

I am writing to apply for the position of
Assistant Professor in Earth Sciences
at the Department of Earth Sciences of Durham University.
I am currently a Visiting Research Scholar at the University of Hawaii at Manoa in the
Department of Earth Sciences.
My background is in geophysics with a focus on gravity and magnetic inverse problems and
scientific computing.

Before coming to Hawaii, I worked for three years as Assistant Professor at the School
of Geology of the State University of Rio de Janeiro (UERJ), Brazil, where I gained
valuable experience in student supervision, teaching at the undergraduate level, and
curriculum development.
At UERJ, I had the opportunity to design and implement three new undergraduate classes:
two geophysics classes for the Geology program, one on gravity and magnetics and another
on exploration seismology, and a programming and numerical methods class for the
Oceanography program.
My strategy for developing these classes was based on current research in active
learning, which has been shown to lead to higher retention and deeper understanding of
the subject matter.
The students used their time in class to work on practical exercises and projects, most
of which involved interactive computational tools that I developed using Python and
Jupyter notebooks (a web-based literate programming tool with support for interactive
graphs and animations).
Learning was achieved through guided experimentation using these computational tools.
In the seismology modules, for example, the students used Jupyter notebooks to generate
animated simulations of seismic wave propagation which they could modify.
The programming class was implemented using the Github website and their Github
Classroom platform, where students were automatically assigned projects, submitted their
work, and receive feedback.
These technologies allowed me to manage a project-based class with over 70 students.
The active learning approach was well received by the students, as evidenced by a
teaching award given to me by the geology program's graduating class of 2016.
With my background in geophysics and my experiences at UERJ, I feel confident in my
ability to teach at a high level and outside of my immediate expertise, including
mathematics, geoinformatics, and introductory modules across the department's courses.


Reasearch:
* outputs
* plan enhances department plans
* Research leadership
* PhD supervision
* Research impact beyond my institution
* income generation (participation in successfull research project proposals)
* connections

My research focuses on gravity, magnetic, and geodetic data processing and inversion.

A majority of my published works are on the development of novel methodologies and the
open-source software that implements them.

Through the development of open tools, I have developed collaborations with researchers
in Italy, Argentina, China, Canada, and the US.

My most widely used software is the Tesseroids program for forward modeling
gravitational fields of spherical prisms.
It has been used in the generation of data products from the ESA's GOCE satellite.

I supervised an undergraduate student on a research project and I am currently
co-advising a PhD student (Santiago Soler) from the Universidad Nacional de San Juan,
Argentina, who recently submitted his first paper to the \textit{Geophysical Journal
International}.

I have developed computationally efficient methods for gravity inversion, including
the estimation of 3D structures in depth and Moho relief.
I am particularly interested in using seismological information to help constrain the
gravity inverse problem.
Recently, I have been investigating the application of machine learning techniques in
geophysical data interpolation.

I left my post at UERJ in 2017 because of the aggravated economic crisis and increase
in violence that made life in Rio de Janeiro unbearable.
In Hawaii, I am developing a Python interface to the Generic Mapping Tools (GMT).

Something about reproducibility and the role of open-source.


% Software
To support my research and teaching efforts, I have created the open-source
Python library \textit{Fatiando a Terra}
(\href{http://www.fatiando.org/}{www.fatiando.org}).
The project uses software development best practices, such as version control
and the Github pull request workflow,
automated tests and continuous integration,
and packaging and distribution through the Python Package Index and
conda-forge.
My current work at the University of Hawai'i is to develop a Python wrapper
library for the \textit{Generic Mapping Tools}
(GMT; \href{http://gmt.soest.hawaii.edu/}{gmt.soest.hawaii.edu}),
an open-source software packaged widely used across the Earth, Atmospheric, and
Ocean Sciences to process and visualize geospatial data.
I have also contributed to other open-source projects, all of which can be
accessed through my Github profile
(\href{https://github.com/leouieda/}{github.com/leouieda}).

% Open science
As a proponent of open and reproducible science, I publish all of the
source-code and data for my first author publications through my research
group's Github page
(\href{https://github.com/pinga-lab}{github.com/ pinga-lab}).
To promote reproducible research best practices within the group,
I maintain a template
(\href{https://github.com/pinga-lab/paper-template}{github.com/pinga-lab/paper-template})
which is used for creating new research projects.
I am interested in exchanging information and learning from the experiences of
the Data Science Workflows course of the MDS program.


% Conclusion
I look forward to the opportunity to learn from the experience of the MDS
faculty and to use my geoscience expertise to expand and enrich the program.
I am particularly interested in the Capstone projects, for which I see great
potential for collaboration with the mining and oil and gas industries.

\closing{Thank you for your consideration,}


\end{letter}

\end{document}
