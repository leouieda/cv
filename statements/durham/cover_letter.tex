\documentclass[11pt]{letter}
\usepackage[utf8]{inputenc}
\usepackage{geometry}
\usepackage{graphicx}
\usepackage{fancyhdr}
% Control the font size
\usepackage{anyfontsize}

% Set the color of hyperlinks
\usepackage[colorlinks=true]{hyperref}
\hypersetup{
    pdftitle={Cover Letter},
    pdfauthor={Leonardo Uieda},
    linkcolor=black,
    citecolor=black,
    filecolor=black,
    urlcolor=black
}

\longindentation=0pt

% Redefine the empty style so the front page also has a header.
\fancypagestyle{empty}{ %
    \fancyhf{} % remove everything
    \lhead{}
    %\cfoot{\thepage}
    \cfoot{}
    \renewcommand{\headrulewidth}{0pt}
    \renewcommand{\footrulewidth}{0pt}
    \setlength\headheight{31pt}
}
\pagestyle{empty}

\signature{Leonardo Uieda, PhD}
\address{
    1680 East-West Road, POST 821
    \\
    Honolulu, HI, USA, 96822
    \\
    email: \href{mailto:leouieda@gmail.com}{leouieda@gmail.com}
    \\
    phone: +1 808 428 4521
}

\begin{document}

\begin{letter}{
        Department of Earth Sciences,  \\
        Durham University, \\
        Science Labs, \\
        Durham DH1 3LE
}
\opening{Dear Members of the Search Committee:}

I am writing to apply for the position of
Assistant Professor in Earth Sciences
at the Department of Earth Sciences of Durham University.
I am currently a Visiting Research Scholar at the University of Hawai'i at Manoa in the
Department of Earth Sciences.
My background is in geophysics with a focus on gravity and magnetic inverse problems and
scientific computing.

% TEACHING
Before coming to Hawai'i, I worked for three years as Assistant Professor at the School
of Geology of the State University of Rio de Janeiro (UERJ), Brazil, where I gained
valuable experience in student supervision, teaching at the undergraduate level, and
curriculum development.
At UERJ, I had the opportunity to design and implement three new undergraduate classes:
two geophysics classes for the Geology program
(one on gravity and magnetics and another on exploration seismology)
and a programming and numerical methods class for the Oceanography program.
My strategy for developing these classes was based on current research in active
learning, which has been shown to lead to higher retention and deeper understanding of
the subject matter.
The students used their time in class to work on practical exercises and projects, most
of which involved interactive computational tools that I developed using Python and
Jupyter notebooks (a web-based literate programming tool with support for interactive
graphs and animations).
Learning was achieved through guided experimentation using these computational tools.
In the seismology modules, for example, the students used Jupyter notebooks to generate
animated simulations of seismic wave propagation which they could modify.
The programming class was implemented using the Github website and their Github
Classroom platform, where students were automatically assigned projects, submitted their
work, and received feedback.
These technologies allowed me to manage a project-based class with over 70 students.
The active learning approach was well received by the students, as evidenced by a
teaching award given to me by the geology program's graduating class of 2016.
With my background in geophysics and my experiences at UERJ, I feel confident in my
ability to teach at a high level and outside of my immediate expertise, including
mathematics, geoinformatics, and introductory modules across the department's courses.

% REASEARCH
My research focuses on the application of gravity, magnetic, and geodetic data modeling
and inversion to understand the state of the Earth's crust and how it evolves over time.
I am particularly interested in global scale investigations of crustal structure using
gravity and seismological data,
estimates of mass change in the Arctic and Antarctic from satellite gravity measurements,
and the use of machine learning techniques in geophysics.
A majority of my published works are on the development of novel methodologies to
extract as much information from the available data as possible.
I am also a proponent for openness in the scientific process.
As such, I develop open-source software that implements the methods that I create and
strive to make all data and analysis code for my publications freely accessible,
mostly through my research group's Github organization
(\href{https://github.com/pinga-lab}{github.com/pinga-lab}).
Through my investment in open tools, I have developed collaborations with
researchers in Italy, Argentina, China, Canada, and the USA.
My most widely cited software, \textit{Tesseroids}, has been used in numerous studies and
in the generation of data products for the ESA's GOCE satellite mission.

I left my post at UERJ in 2017 because of the aggravated economic crisis and increased
violence in Rio de Janeiro.
At the University of Hawai'i, I am developing a Python language interface to the Generic
Mapping Tools (GMT), an open-source software widely used in the Earth Sciences.
I am also involved in the core GMT development and modernization efforts.
In collaboration with researchers at the Scripps Institution of Oceanography, I am
developing new interpolation methods for Global Navigation Satellite System (GNSS) data.
This work has been funded as part of a grant from the USA National Science Foundation's
EarthScope program, on which I am co-principal investigator.
I am currently
co-advising a PhD student (Santiago Soler) from the Universidad Nacional de San Juan,
Argentina, on large scale gravity modeling and inversion.
Santiago recently submitted his first paper to the \textit{Geophysical Journal
International} and is currently developing a new method for estimating mass change in
Antarctica from GRACE satellite gravity data.

% Conclusion
Modern research is under pressure from increased volumes of data and stronger
requirements for reproducibility of experimental and computational results.
Through my expertise in robust software development and the analysis of large datasets,
I hope to collaborate across the Department of Earth Sciences to help tackle these
issues.


\closing{Thank you for your consideration,}


\end{letter}

\end{document}
