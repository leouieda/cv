\documentclass[12pt,notitlepage]{article}
\usepackage[left=1.5in,right=1.5in,top=1.5in,bottom=1.5in]{geometry}

\usepackage{graphicx}
\usepackage{url}
\usepackage[utf8]{inputenc}


\begin{document}

\begin{center}
    {\huge Research Plan}
    \\[0.2in]
    {Leonardo Uieda}
    \\[0.1in]
    {\small \today}
\end{center}


As a geophysicist, my ultimate goal is to infer knowledge about the inner Earth and its
processes from surface observations, such as its gravity and magnetic fields.
This inference is an ill-posed inverse problem, to which a solution might not exist or
be non-unique and unstable.
My focus is on tackling the theoretical and computational challenges that arise from
these inverse problems.
The main questions that drive my research efforts are:

\begin{enumerate}
    \item How can we better constrain gravity and magnetic inverse problems?
    \item How can we make more efficient algorithms to deal with increasing data volumes?
    \item What steps need to be taken to make computational geoscience more reproducible
        and accessible?
    \item Which techniques from different fields (e.g., statistics, machine learning)
        can be applied to solve problems in geoscience?
    \item How can multiple types satellite observations be used to constrain the
        movement of masses?
\end{enumerate}

As attempts to address questions 1 and 2, I have developed new gravity inversion methods
(Uieda and Barbosa, 2012 and 2017) and collaborated on the development of others
(Oliveira Jr et al., 2015; Zhao et al., unpublished).
To enable modeling on global and continental scales, I improved upon existing forward
modeling methods using spherical elements (Uieda et al., 2016).

% Inversion
My first attempt to tackle questions 1 and 2 resulted in Uieda and Barbosa (2012), in
which we developed an efficient 3D inversion method for gravity gradient data.
At the time, new instrumentation spurred the interest in gravity gradiometry and
research groups struggled to develop inversion methods for this new type of data.
In Uieda and Barbosa (2017), we developed another method to estimate the depth of the
crust-mantle interface from gravity data using seismological results as constraints.
In order to apply this method to large areas, we used spherical elements (tesseroids)
and the forward modeling developed in Uieda et al. (2016).
Currently, the work on forward modeling with tesseroids is being carried out by
Santiago Soler, a PhD student whom I am co-advising.
He developed a formulation for modeling tesseroids with depth-variable density, which
is currently undergoing peer-review.

% Processing and machine learning


% Software and reproducibility

Modeling across geophysics

Global to local scale gravity

Time-lapse for mass balance.

Started working on antartica GRACE with Santiago

Processing GPS and InSAR with Dave (prof here)

Joint inversion of time lapse gravity and insar

Joint inversion of gravity and seismic (prof here)

Joint inverison of gravity, seismic, topography for Moho

Implications of Moho depth and Curie surface for geothermal

Software to support openness

Collaborations with outside people. Bo and Guangdong. Santiago. Carla.

Continue with GMT dev.

GMT has huge impact.

Modernizing and bringing to Python.

Expand seed to magnetic with pinga

Reproducibility: try new things, keep recording things on git, open software is required

Impact of software is wide. Tesseroids key to some GOCE products. Working on using for
GRACE.

Openness is at the core.

Communication through web technologies. Websites and interactive computation to promote
awareness. Collaborate with prof on hazards communitation.

Grants. Which ones would I try? NERC for GRACE inversions antartic and arctic.

NERC for environment (GRACE stuff)

EPSRC for software (maybe training in software design and best practices for scientists)







\textbf{Software}:
Methodological development requires
much prototyping and iteration.
%
Thus,
a researcher needs
a flexible environment
and a large collection of tools
for experimentation.
%
The approach I have taken is
to develop an open-source library
called Fatiando a Terra\footnote{\url{http://www.fatiando.org}}
that collects the basic tools
required for building an inversion method.
%
The library is implemented in Python,
a dynamically typed interpreted language
known for its simplicity
and large ecosystem of scientific libraries.
%
Fatiando is developed in the
open\footnote{\url{https://github.com/fatiando/fatiando}}
with the help of a growing, though yet small,
developer community.
%
I use it as the basis for
all of my research projects
as well as for teaching geophysics.
%
My first open-source project
was \textit{Tesseroids}\footnote{\url{http://tesseroids.leouieda.com/}},
a collection of command-line programs
for forward modeling gravitational fields
using spherical prisms.
%
It is written entirely in C
and is my most widely used software project to date.


\textbf{Inverse problems}:
During my graduate studies,
I have developed two novel inversion methods
for gravity data.
%
The first is a 3D gravity gradient inversion
using a heuristic algorithm
that grows the solution from starting ``seeds''.
%
The method is computationally efficient
and can handle problems with millions of unknowns
on a mid-range laptop computer.
%
The second method inverts gravity data
to estimate the relief
of the crust-mantle boundary
using a spherical approximation for the Earth.
%
A common theme of my research has been
to use, adapt, and improve upon
highly efficient algorithms
to solve the problems of geophysical inversion.


\textbf{Reproducibility}:
Computational experiments
are difficult, if not impossible, to reproduce
without access to the code used to generate them.
%
I attempt to tackle
some of these issues
on my own research
by making all source-code and data
(as much as legally possible)
from my own publications
available in public repositories\footnote{For example \url{https://github.com/pinga-lab/paper-moho-inversion-tesseroids}}.
%
I hope to train my students
in these practices from the start
and provide guidance for others to do the same.
%
Though I cannot claim
to generate fully reproducible results,
I have been refining this process over time
and hope that my efforts will contribute
to advance the reproducibility of our science.

\end{document}
