\documentclass[12pt,notitlepage]{article}
\usepackage[left=1.5in,right=1.5in,top=1.5in,bottom=1.5in]{geometry}

\usepackage{graphicx}
\usepackage{url}
\usepackage[utf8]{inputenc}


\begin{document}

\begin{center}
    {\huge Research Plan}
    \\[0.2in]
    {Leonardo Uieda}
    \\[0.1in]
    {\small \today}
\end{center}


As a geophysicist, my ultimate goal is to infer knowledge about the inner Earth and its
processes from surface observations, such as its gravity and magnetic fields.
This inference is an ill-posed inverse problem, to which a solution might not exist or
be non-unique and unstable.
My focus is on tackling the theoretical and computational challenges that arise from
these inverse problems.
The main questions that drive my research efforts are:

\begin{enumerate}
    \item How can we better constrain gravity and magnetic inverse problems?
    \item How can we make more efficient algorithms to deal with increasing data volumes?
    \item What steps need to be taken to make computational geoscience more reproducible
        and accessible?
    \item Which techniques from different fields (e.g., statistics, machine learning)
        can be applied to solve problems in geoscience?
    \item How can multiple types satellite observations be used to constrain the
        movement of masses?
\end{enumerate}

As attempts to address questions 1 and 2, I have developed new gravity inversion methods
(Uieda and Barbosa, 2012 and 2017) and collaborated on the development of others
(Oliveira Jr et al., 2015; Zhao et al., unpublished).
To enable modeling on global and continental scales, I improved upon existing forward
modeling methods using spherical elements, or tesseroids (Uieda et al., 2016).
However, all of these new methodologies would not be immediately useful to other
researchers without a software implementation.
Throughout my career, I have developed and maintained open-source software that
implements the methodologies that I create and any supporting tools required for my
research.
Making these projects open-source and freely available to others allows me to share all
facets of my work: the manuscript explaining the method, the source code that implements
it, and the data analysis pipeline used to generate published results.
This freedom is essential to achieve reproducibility (or at least replicability) in
science.

In the future, I plan to expand the development of my open-source projects by attracting
and training new contributors.
The impact of these software tools is wide and enables synergistic interactions with
other researchers through the sharing of a common digital infrastructure.
I will also continue my collaborations with researchers at the University of Hawai'i
and my involvement in the Generic Mapping Tools (GMT) project.
I am currently developing an interface between GMT and the popular Python language,
for which there is a lack of feature rich mapping tools.
Building communities around open-source projects requires the creation of a pool of
researchers trained in software development best practices.
As part of my teaching efforts, I will investigate funding sources for running workshops
on software development and basic computational literacy for scientists.
A possible funding source is the EPSRC, which has funded this type of activity in the
past.
I envision collaborations with Mathias and others from the department in these
activities, which would certainly benefit the entire program.
These skills are in high demand in both academia and industry, which makes them
invaluable to graduate students and early career scientists.

For my research on inverse problems, I am investigating the use of gravity data from the
GRACE satellite in conjunction with tesseroids to better estimate mass loss from the
melting glaciers in Antarctica and the Arctic.
I plan to investigate the use of space geodetic data, namely InSAR, to help constrain
the gravity inversion.
Collaborations with others at the Department of Earth Sciences (e.g., Walters) and the
Institute of Hazard, Risk and Resilience would be highly valued given the complexity of
the issue.
Furthermore, I will continue my work on large scale gravity inversion with a focus on
integrating isostatic, heat flow, and seismological data into the solution.
Gravity data can cover large areas at a small cost but require other data sources to
constrain the depth dimension.
A common choice is to integrate seismological data into the inversion, for which I would
welcome collaborations with Hobbs, Peirce, and others.

I will also continue my collaborations with researchers at the Scripps Institution of
Oceanography on interpolation methods for GNSS and InSAR data. The results obtained
through this research can be expanded for processing airborne and satellite gravity and
magnetic data, which is one the areas of expertise of my collaborators from Brazil.
I am particularly interested in applying techniques from the field of machine learning
to these problems.

Finally, openness is at the core of my approach to teaching and research.
I will continue to leverage open web technologies and interactive computational tools to
promote the public understanding of science and programming.
I advocate for open-access publishing and hope to collaborate with Wadsworth to bring
the model implemented in the Volcanica journal to my own field of geophysics.




\end{document}
